\documentclass[12pt]{article}
\usepackage[T1]{fontenc}
\usepackage{inputenc}
\usepackage{amsmath}
\usepackage{amssymb}
\usepackage{amssymb}
\usepackage{indentfirst}
\usepackage[no-math]{fontspec}

\usepackage{geometry}
\geometry{a4paper,scale=0.9}

\setlength\parindent{2em}
\linespread{1.25}

\usepackage{xeCJK}
\xeCJKsetup{PlainEquation=true}
\xeCJKsetup{CJKmath=true} 
\setCJKmainfont[ItalicFont={楷体}]{Noto Serif CJK SC}
\setCJKmonofont{SimSun} 
\setCJKsansfont{黑体}    
\usepackage{fourier} % set math font
% \usepackage{mathptmx} % set math font
\everymath{\displaystyle}
% \setmainfont{Palatino-Roman}


% \usepackage[siunitx,RPvoltages,european]{circuitikz}  % 使用英式符号
% \def\killdepth#1{{\raisebox{0pt}[\height][0pt]{#1}}}  % 对齐命令,保持元器件的基线一致

\usepackage{tikz}
\usetikzlibrary{datavisualization}
\usetikzlibrary{angles}
\usetikzlibrary {arrows.meta}
\usepackage{xcolor}

\renewcommand{\,}{\ \text{,}}
\renewcommand{\.}{\ \text{.}}
\newcommand{\proofover}{\hfill $\square$}

\begin{document}


\section{为什么要给出线性代数的几何意义}

作为一名工作十多年的电子工程师,作者想再提高自己的专业水平时,深感数学能力的重要.随便打开一篇论文或一部专著,满纸的向量、矩阵、微分方程词汇扑面而来.竭力迎头而上,每每被打得灰头土脸、晕头转向.我天生就不是搞数学的?我的智力有问题吗?在高校、科研院所或大型企业里工作,可能还有专家、教授级的导师指引着进行理论上的研究,但大部分的大学生一毕业就进入了各类中小企业,在这些企业里的研发工作,基本上都是低级模仿,想弄点理论研究常无从下手,感觉大学是白上了.找点资料看看吧,大多是数学公式满篇,解释抽象模糊、思维之断崖重重.茫然之中仿佛看到专业作者们高深莫测、神龙无尾、高傲自去的背影.

太失望了,太伤自尊了.扭头看看周围的同行,莫不雷同.大多的工程师们靠经验来工作,经验靠时间或试验来积累.数学应用的层次多是高中水平.也有硕士博士级的牛人,但也少见把数学工具在工作中应用的得心应手、手到擒来的.

数学工具在科技实践中缺失的严重,导致我们的企业科技创新能力的严重不足.普遍现象,绝对的.

返回来想一想,我的智力应该没问题,重点大学都毕业了,能有多严重的问题?所有的工程师们、大学毕业生们的智力也没问题.问题是大家没把数学学好,没有真正掌握它.

\begin{quote}
屌丝的声明:

数学绝顶高手和天才们不在俺说的范围之内,对我等来说,他们是极少数的一小撮的火星人,对他们只能顶礼膜拜,不敢评论.不过拜完之后有点小嘀咕:为何钱学森还讲中国没有大师呢?为什么数学总考一百分的天才也难以成为大师?
\end{quote}

为啥没有在四年的大学阶段学好``线性代数''呢?要知道,学生是通过高考百里挑一录取的,智力应是足够正常的.思来想去,得到几个原因:教材编的大多不好,老师教的大多乏味,学生大多有些偷懒.学生偷懒原因不只一个,但我觉得主要一个原因是他们学起来困难-因为他们大多不知道这些内容有啥用,概念为啥这么叫,定理为啥那样推,老师为啥像刘谦的魔术一样七推八导就证毕了------郁闷多了导致了无语的偷懒.

太多的为啥了.既然错不在学生那就是老师的问题了?其实老师也很委屈:教学大纲要求在几十个学时学习如此多的内容,不填鸭行吗?在如此短的时间内讲完就不错了,哪里还有时间给你释疑解惑------韩愈定义的传道授业解惑的师道中的解惑被迫取消了,自己悟道吧.殊不知,惑之不解则道难传而业无授也.

嘿,错也不全在老师那里.错在哪里?找来找去,似乎有一个大家都可以责备而少有抗议的地方,就是教材不够好,因为学生靠看教材自习的话也困难重重:工科教材太简单的话看起来不深入,理解很肤浅,似懂非懂;专业数学书确是详细,但还是太抽象,一个新概念接着一个新概念从天而降,也让人发晕.

有人说,线性代数是培养大学生的抽象思维用的,就是让你``发晕''的.还有人说,工科的线性代数课程要求低,只是让你掌握一个数学的计算工具罢了,你会做题通过考试就可以了,干吗思前想后的.

有点欺负人不是?俺学一门学科真心就是要掌握它的真谛嘛,不发晕才是真掌握,真掌握了才算是学会了抽象思维.再说了,会做题也不等于真正掌握了这个数学工具,只有在实际应用中对它建模才能应用,数学建模的过程正是由具体到抽象的过程,但是不发晕才能抽象啊----问题又回来了.

看来还是要真掌握真明白才是正道.真明白就是把自己的疑问一一解决,找书看去!

到附近的大学图书馆看看,哇塞,一行行、一列列的教材琳琅满目、浩如烟海.名字叫``线性代数''的教材足有一千多册.

随便打开一本《线性代数》看看,跟十八年前的教材内容几乎一样:简单、生硬.最多就是把一点解析几何塞进去,水油不溶,感觉不像一本书(线性代数和高等几何的理论框架是不同的,线性代数只讲向量及其向量集合的几何空间图形,而高等解析几何是向量(如法向量)和点集图形混为一谈,目的是坐标点的几何分析.请参考本书5.14.2节里面的对偶空间的分析内容),疑问还是没得到解决.再打开一本看看,内容还是那个内容,疑问还是那个疑问.....

当浏览到第500本的时候(有点夸张哈),终于看到了我那个问题的答案了.长出一口气后我又陷入了郁闷之中.要知道,我至少有十打问题要解决呀,老天.

真是皇天不负有心人!眼前一亮,一本老外所编教材图文并茂的内容一下子照亮了俺那已黯然的心灵之窗(俺不想崇洋),我兴奋得一阵眩晕.要知道,老外的教材大都是引入了当代科技的典型应用案例的,代表了本学科的国际新潮流,具有超强的问题杀伤力.

(注:当然,国内也不乏优秀的线代教材,比如陈怀琛的《线性代数实践及MATLAB入门》,侧重工程实际和计算工具使用;李尚志的《线性代数》教材,理论讲解全面细腻,试图将几何代数熔于一炉,但仍稍嫌抽象,专业性强.另外,俺发现一个现象,现在的教材多不如十几年前的老教材,如张远达的《线性代数原理》讲解就很牛.)

大学图书馆里的读者很多,朝气蓬勃的青春学子们在图书馆里做作业.我很羡慕他们这一代:在开放的图书馆里,学生们在书架前可以随意地浏览、挑选适合自己的纸质或电子版读物.要知道,当年我就读的大学图书馆是闭架的,每每借书要查半天小卡片,查完填好借书单交给工作人员大多得到两个结果:要么书被借完了,要么借的书不合适.而且还没有这么多的引进教材供参考,自学的效率大打折扣.

扯来扯去,千言万语汇成一句话:什么样的线性代数学习资料较好,较适合中国学生?\\
我想,本子的物理尺寸要越薄越好,内容要越通俗易懂越好.

书本越薄,大家学习的信心越强:小样,这么点厚度还搞不定你,看,信心先有了.

如果只是容量精简了还不行,因为内容不全面,讲述不到位使得考试的时候受打击,工作中更受打击.如当年我学的《线性代数》课本是同济编的,内容是精简到家,千锤百炼,没一句废话,超薄.死记硬背,看似搞定了,实际是圆图吞枣,似懂非懂.

如何通俗易懂还不能多说,我一直认为,加上几何意义或者物理意义啥的(再加上点当代的经典应用就更美了),一步到位搞定.

这就是本书的由来,也是本书的目标.

方向有了,具体如何编写呢?模仿一下科学大德牛顿的口气:

\textbf{从线性代数书籍之浩瀚海洋的沙滩上(还没有更高的能力去远洋),用一双自己的眼睛,寻找到了一个个闪亮的小珍珠,一片片如玉的小彩贝,然后细细地擦拭、打磨,拂去沙尘,使它们重放光彩,用一根几何意义的锦丝,穿就了这本《线性代数的几何意义》的项链,献给热爱思考、痴迷于创造的人们.}

呵呵,自不量力,终极目标而已,但意思还是有了.

\section{重要的几何直观意义}

在学习中,一旦碰到较抽象难懂的新概念或定理,如何搞定?几个办法:一个是看推导过程,推导可以加强你相信它的信心并连通你原有的知识体系.如果推导把你弄昏了,最好弄懂它的几何意义或物理意义.

几何意义或者讲几何解释会和人们看到的平面和空间中物体几何外观联系起来,几何上说得通,物理上也就说得通,因为几何意义和物理意义本质上是一回事,新概念的几何意义通了,新概念就会和读者大脑中的既有经验或知识网络连通,一下子就``懂了'',满心欢喜的,原来是这么一回事.

有的哥们不太赞同几何意义和物理意义本质上是一回事.仔细想想吧,几何是描述现实空间的,而现实空间又是由物理规律所决定的;几何里蕴含着物理,物理决定着几何结构的存在.而且著名的进化论说过,物竞天择,适者生存.这个道理不光在说生物界,整个宇宙都符合进化论——只有适合物理规律的物体才能生存.直接说吧,$n$亿年过去了,不符合物理规律的物质几何空间早就灭亡了.有点武断?提几个活生生的例子吧,杨振宁是物理学家,他搞的规范场是纯粹的物理理论,后来他突然发现,规范场正是微分几何学科里面的微分流形纤维丛上的联络.因此微分几何大师陈省身感慨地说``数学家和物理学家所研究的,只是一头大象的不同部分'',如果你再不信物理和几何是一回事,就想想爱因斯坦,想想他的相对论,相对论把宇宙包括牛顿的世界直接几何化了,用几何直接解释物理现象.

\textbf{真理总是简单的和直观的},一位先贤说,不管多么复杂高深的数学理论,总有其直观的背景,不管多么繁难深奥的定理,其证明总有一个简单而直观的中心思想.几何图形能以其生动的直观形象给人留下深刻的印象.可以这样说,在数学中再没有别的什么东西,能比几何图形更容易进入人们的脑海了.

从宏观上看,一种数学理论(包括它的主要概念和方法)往往都有其直观的背景,它们或者是从对某些特殊的事例的观察分析中得到的,或者是直接从几何图形中看出的,或者是从已有的结果类比联想引来的,从几何直观上分析问题的能力,首先是指对于一种数学理论能``洞察其直观背景''.对于它是如何被发现的或如何形成的作出合理的解释或猜测.

一句话,一部皇皇巨著的理论特别是抽象的数学理论的核心常常可以从几何意义的角度得到解释.

从微观上看,国外的数学教育家波利亚曾经说:``一个长的证明常常取决于一个中心思想,而这个思想本身却是直观的和简单的''.因此,从几何直观上分析问题的能力,也包括找出证明中的那个关键的简单而直观的思想,也就是像希尔伯特所要求的,能透过概念的严格定义和实际证明中的推演细节,``描绘出证明方法的几何轮廓''.

大师庞加莱和阿达玛关于数学领域的发明创造的观点也认为,数学创造发明的关键在于选择数学观念间的``最佳组合'',从而形成数学上有用的新思想和新概念,而这种选择的基础是``美的直觉''.在这种美的直觉中,也就是在追求某种对称性、和谐性、统一性、简洁性和奇异性当中,以及在某种联想、猜想、假设及非逻辑思维中,几何直观具有头等重要的意义.

事实上,很多数学家都是先利用几何直观猜测到某些结果,然后才补出逻辑上的证明的.这正如我国拓扑学家张素诚先生所说的:对数学中的许多问题来说,``灵感''往往来自几何,表达的简洁靠代数,计算的精确靠分析.

革命年代的数学偶像华罗庚也曾吟过``数缺形时少直观,形少数时难入微;数形结合百般好,割裂分家万事休''.

嘿嘿,看看上面的数学上的历史牛人的观点,几何形象直观的意义何等重要.其实,大家都知道几何意义的重要,我们在小学和中学的学习阶段,老师也常常讲一些抽象概念所对应的几何意义,为何到了大学我们的大脑就一下子高度抽象起来了?把形象扔得远远的,像瘟疫一样躲着它?目的是训练抽象思维,最终实际结果呢?不可否认,大学毕业后大家确实是抽象了,抽象得只会夸夸其谈讲理论不会干具体活了.既然你具体的活儿不会干那干脆就专搞抽象的理论去嘛,结果也搞不了,为啥?只会做做过的抽象的数学题不会发明创造,没学会真正的抽象,真是越抽象越糊涂.

几何意义和几何图形有点区别,我觉得几何意义应该是直观的、空间的、形象的、感性而可想象的含义,当然重点包括几何图形.如果只是些三角形、圆锥曲线、三维曲面类的几何图形远不能够让我们把握越来越抽象的数学,我们还应有点莫比乌斯带、克莱因瓶、不动点类的东东来激发我们的想象力.强调几何意义就是别忘了我们数学的想象力,具有了数学理论的想象力才会在我们的工程实践中运用她,体会她,发展她.

我觉得,抽象和形象是相辅相成,缺一不可的.套用马列主义辩证法的说法就是由形象而抽象,再由抽象到形象,人的知识结构螺旋架才能旋转而上,达到越来越高的知识巅峰.

\section{如何使用这本书}

拼命阐述几何直观在数学学习中的重要意义,但这并不意味着可以否定逻辑推理论证的重要作用.实际上,单纯地依据直观而导致错误的数学例子真是数不胜数.概念或定理的几何直观解释,往往并不等同于原来的概念或定理.运用几何直观可以帮助我们猜想,但猜想并不能代替证明,只有经过一步步严格的逻辑论证以后,才算给出了证明.

形象(或直观)和抽象本来是一切科学的两面.只是近年来过分强调了抽象思维能力的训练而忽视了几何意义的解释.反过来,我们不能只强调几何意义而丢掉了计算和推导.因此建议读者:

\begin{itemize}
\item
  从几何或物理意义入手,轻松而迅速理解和把握线性代数的基本概念和定理的几何本质,建立对线性代数的感性认识,具备理解复杂及抽象数学的基础.
\item
  然后,再回到现在学的抽象的线性代数的教材,通过适量的习题训练,短时间内构筑个人的线性代数知识体系的``向量空间'',巩固解决具体问题的动手能力.此时,\textbf{具体与抽象一体,理想与现实齐飞}.您,已经成为线性代数的高手和大牛.
\end{itemize}

我希望这本书对于高中或大学的初学者、学习过线性代数的、考研生、工程师、教师和任何喜欢数学想象的人都有帮助.具有一些线性代数基础的读者可能更喜欢这本书,因为书里一直没有回避公式和必要的推导.我不喜欢把她写成一个没有公式的大众科普读物,像霍金的科普书那样做只能让读者有一个仍然空洞的字宙感性而不能真正深入数学的内部.

注:本文中,几何意义和几何解释的文字意思没有根本区别,一般对于数学概念的对应的几何图形而言称为几何意义,而对运算、变换的过程可对应几何图形的变化过程称为几何解释.

\begin{flushright}
    \begin{minipage}[c][5cm][c]{3cm}
        \centering
        作者\\
        2015年3月
    \end{minipage}
\end{flushright}

\end{document}