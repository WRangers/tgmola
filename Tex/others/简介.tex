\chapter*{内容简介}

本书是为热爱科学、热爱数学的思考者准备的!

本书是为喜欢把握科学脉搏探寻科学真谛的人准备的!

本书是为不愿被动地接受知识而奋力挣扎的学子准备的!

本书使用向量的概念对国内高校工科“线性代数”的课程内容进行了较全面的几何分析.从向量的几何意义开始,分别讲述了向量组、向量空间、行列式、矩阵、线性方程组和二次型的几何意义或几何解释,其中不乏重要概念的物理意义的解释.大量的代数概念及定理的几何意义的解释也可以使它成为您学习线性代数的参考手册.

本书大多为作者的原创,比如叉积的物理意义,克莱姆法则、雅可比矩阵、相似/合同矩阵、转置矩阵/对偶、矩阵乘积的行列式等系列概念的几何意义等,应用方面如使用矩阵分析的方法分析电子振荡器的工作原理等.

本书图文并茂,思路清晰、语言流畅,概念及定理解释得合理、自然,适合有一定线性代数基础的大学生阅读.由于本书是直接根据线性代数课程的要求进行解释的,同时具有通俗性、科普性,因而也适合初学者和自学者使用.
