\documentclass[12pt]{article}
\usepackage[T1]{fontenc}
\usepackage{inputenc}
\usepackage{amsmath}
\usepackage{amssymb}
\usepackage{amssymb}
\usepackage{indentfirst}
\usepackage[no-math]{fontspec}

\usepackage{geometry}
\geometry{a4paper,scale=0.9}

\setlength\parindent{2em}
\linespread{1.25}

\usepackage{xeCJK}
\xeCJKsetup{PlainEquation=true}
\xeCJKsetup{CJKmath=true} 
\setCJKmainfont[ItalicFont={楷体}]{Noto Serif CJK SC}
\setCJKmonofont{SimSun} 
\setCJKsansfont{黑体}    
\usepackage{fourier} % set math font
% \usepackage{mathptmx} % set math font
\everymath{\displaystyle}
% \setmainfont{Palatino-Roman}


% \usepackage[siunitx,RPvoltages,european]{circuitikz}  % 使用英式符号
% \def\killdepth#1{{\raisebox{0pt}[\height][0pt]{#1}}}  % 对齐命令,保持元器件的基线一致

\usepackage{tikz}
\usetikzlibrary{datavisualization}
\usetikzlibrary{angles}
\usetikzlibrary {arrows.meta}
\usepackage{xcolor}

\renewcommand{\,}{\ \text{,}}
\renewcommand{\.}{\ \text{.}}
\newcommand{\proofover}{\hfill $\square$}

\begin{document}


本书原想写得又薄又精炼,没想打印下来还挺厚的,删除谁都有点可惜,就请诸位将其权当线性代数概念的几何意义手册参阅.再说了,如果东西写得清晰而详尽的话,读起来就感觉不到厚度了.因此,后期的想法是不求本子薄,只求理儿明,不知道做到没有.

本书的第一稿倒是写得不太慢,从儿子出生(07年9月)开始写,一年多就弄完了.放在网上折腾了两年,后来网友们反应热烈纷纷建议出版,就联系到了母校——西安电子科技大学出版社.在毛红兵编辑的悉心指导下开始着手修改文稿.

常常在安静的深夜,守着一叠厚厚的处处用红笔勾画过的稿子,俺一边修改着稿子,一边感叹着出书的不易,一边感念着编辑老师的辛劳.

排版修改太费时间了,几乎全部图片、公式至少重打了3遍以上,工作一忙有时一两个月都没有动一个字.本书的姗姗来迟令俺愧对网友的热望,有时都不敢再到网上去浏览网友的热切留言.

感谢毛红兵编辑的耐心和信任,感谢诸位网友的批评指正,感谢辛勤的合作者谢聪博士、胡翠芳老师,感谢家人在背后默默的支持让我有闲情逸趣来写这本纯粹的“闲书”.事项进行到这总算给热情的读者有点交待了.

最后,一个老生常谈却必须要说的是:作者业余的数学水平有限,缺憾谬误和笔误一定很多(虽然已修正了很多),请诸位亲们勿怕麻烦,把这些脸上雀、肉中刺指点出来.俺无它以报,当赠书为念.

联系邮箱:renguangian@126.com.

\begin{flushright}
    \begin{minipage}[c][5cm][c]{3cm}
        \centering
        作者~任广千\\
        2015年3月
    \end{minipage}
\end{flushright}

\end{document}